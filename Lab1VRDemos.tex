\documentclass[11pt]{article}
\title{Lab 1: VR Demos}
\author{Joe Tortorello}
\date{21 March 2018}
\begin{document}
\maketitle
\section{HTC Vive}
\subsection{Accounting}
\textit{Accounting} is a VR demo where the player is hired for a new accounting job, and inadvertently gets stuck in a virtual world, and must find out that her or she must commit suicide to get out. While rather crass, \textit{Accounting} still serves a good example of a game that has a very low level of realisticness, yet is still engaging. Notably, it does not try to trick the user into thinking that he or she is in a real experience at all: it openly acknowledges that it is a virtual reality experience. The user's ``hands'' are replaced with the arrow and pointing finger shapes familiar from the mouse interface on Microsoft Windows, and the user must repeatedly place virtual HMDs on his or her virtual head to progress through the game. One of the things that I did not like was the lack of visual feedback for the hand controls. The cursor-like hand shapes did change from an arrow to a pointing finger to indicate an object in the virtual world that could be interacted with, but there was no change to show when the user was actually interacting. This caused me to repeatedly grab ``next to'' an object when I was trying to pick something up.
\subsection{Google Earth VR}
Google Earth VR was a bit different than some of the other demos because it was the most realistic in some senses, while not realistic at all in others. Google Earth VR was just a simple VR implementation of the existing Google Earth and Google Street View applications.
\subsection{Surgeon Simulator}
\subsection{VR Chat}
\section{Microsoft HoloLens}
\subsection{AR Paint}
\subsection{Microsoft Edge}
\section{Oculus Rift}
\subsection{Fruit Ninja}
\subsection{Ritchie's Plank Experience}
\end{document}