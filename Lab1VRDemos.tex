\documentclass[11pt]{article}
\usepackage[margin=1in]{geometry}
\title{Lab 1: VR Demos}
\author{Joe Tortorello}
\date{21 March 2018}
\begin{document}
\maketitle
\section{HTC Vive}
\subsection{Accounting}
\textit{Accounting} is a VR demo where the player is hired for a new accounting job, and
inadvertently gets stuck in a virtual world, and must find out that her or she must commit
suicide to get out. While rather crass, \textit{Accounting} still serves a good example of
a game that has a very low level of realisticness, yet is still engaging. Notably, it does
not try to trick the user into thinking that he or she is in a real experience at all: it
openly acknowledges that it is a virtual reality experience. The user's ``hands'' are
replaced with the arrow and pointing finger shapes familiar from the mouse interface on
Microsoft Windows, and the user must repeatedly place virtual HMDs on his or her virtual
head to progress through the game. One of the things that I did not like was the lack of
visual feedback for the hand controls. The cursor-like hand shapes did change from an
arrow to a pointing finger to indicate an object in the virtual world that could be
interacted with, but there was no change to show when the user was actually interacting.
This caused me to repeatedly grab ``next to'' an object when I was trying to pick
something up.
\subsection{Google Earth VR}
Google Earth VR was a bit different than some of the other demos because it was the most
realistic in some senses, while not realistic at all in others. Google Earth VR was just a
simple VR implementation of the existing Google Earth and Google Street View applications.
\subsection{Surgeon Simulator}
Surgeon Simulator was also similar to \textit{Accounting} in that it employed a
cartoon-like interface, and did not try to be very realistic. Compared to the desktop 
release of \textit{Surgeon Simulator}, which I have also played, I liked the way that 
this one used the controls. In the desktop version, the surgeon's hand is mapped to the 
A, W, E, R and Space keys on the keyboards to correspond to the fingers on the left hand 
of the player. The version for the HTC Vive employs the grip control on the controller, 
which was a much more natural way of controlling the game for me. There was not much that 
I did not like about this game, but it did annoy me that to start it you had to step into 
a glowing circle that would usually end up right behind you.
\section{Microsoft HoloLens}
\subsection{Holograms}
Microsoft's \textit{Holograms} program for the HoloLens was an excellent demo of what the 
device is capable of. The holograms tracked extremely well against the surrounding room,
and some actually seemed pretty realistic. Most of the holograms were not interactive,
which allowed this app to avoid a major problem that the HoloLens has: it is not very
responsive to input, and the input is uncomfortable. Over time, I got better at getting
the HoloLens to recognize my clicks, but holding your hand in the position needed to do so
gets tiring quickly.
\subsection{Microsoft Edge}
A web browser may seem like a rather mundane thing to experience in virtual reality, but
the way that the HoloLens allowed it to be placed upon a surface in the room and used was
actually quite novel. I particularly enjoyed that it offered the ability to go on a web
site like YouTube and watch videos, as if you were looking at a virtual television
screen. As mentioned before, inputting to the user interface was difficult, though.
Particularly bad was the fact that you had to enter text by individually clicking each
letter you wanted to type on a virtual keyboard.
\section{Oculus Rift}
\subsection{Fruit Ninja}
\textit{Fruit Ninja} was much like it's counterpart for mobile phones. The player had to
utilize the motion controls like swords to chop through various fruits that shot out of
the ground. This was a very fast-paced game, and it was exciting to play because it
was able to keep-up very well with movements by the user. This game certainly provided
much immersion despite having very a non-realistic style, as the user is forced into a
situation where he or she has to focus fully on the task at hand: chopping as much fruit
as possible. My only complaint about this game was that it does not allow the user to walk
around the playing field, he or she is stuck right in front of where the fruit shoots out
and it seems a little too close-up to me.
\subsection{Ritchie's Plank Experience}
\textit{Ritchie's Plank Experience} provided a unique experience that one would not get to
actually perform in real life (well, at least not more than once): jumping off the top of
a building. The animations, sounds, and the integration of an actual plank that was so
accurately mapped to the virtual plank really made me feel very present within the demo.
I really have no complaints about the execution of this application.
\end{document}
