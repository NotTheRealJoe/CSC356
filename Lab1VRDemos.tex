\documentclass[11pt]{article}
\title{Lab 1: VR Demos}
\author{Joe Tortorello}
\date{21 March 2018}
\begin{document}
\maketitle
\section{HTC Vive}
\subsection{Accounting}
\textit{Accounting} is a VR demo where the player is hired for a new accounting job, and
inadvertently gets stuck in a virtual world, and must find out that her or she must commit
suicide to get out. While rather crass, \textit{Accounting} still serves a good example of
a game that has a very low level of realisticness, yet is still engaging. Notably, it does
not try to trick the user into thinking that he or she is in a real experience at all: it
openly acknowledges that it is a virtual reality experience. The user's ``hands'' are
replaced with the arrow and pointing finger shapes familiar from the mouse interface on
Microsoft Windows, and the user must repeatedly place virtual HMDs on his or her virtual
head to progress through the game. One of the things that I did not like was the lack of
visual feedback for the hand controls. The cursor-like hand shapes did change from an
arrow to a pointing finger to indicate an object in the virtual world that could be
interacted with, but there was no change to show when the user was actually interacting.
This caused me to repeatedly grab ``next to'' an object when I was trying to pick
something up.
\subsection{Google Earth VR}
Google Earth VR was a bit different than some of the other demos because it was the most
realistic in some senses, while not realistic at all in others. Google Earth VR was just a
simple VR implementation of the existing Google Earth and Google Street View applications.
\subsection{Surgeon Simulator}
Surgeon Simulator was also similar to \textit{Accounting} in that it employed a
cartoon-like interface, and did not try to be very realistic. Compared to the desktop 
release of \textit{Surgeon Simulator}, which I have also played, I liked the way that 
this one used the controls. In the desktop version, the surgeon's hand is mapped to the 
A, W, E, R and Space keys on the keyboards to correspond to the fingers on the left hand 
of the player. The version for the HTC Vive employs the grip control on the controller, 
which was a much more natural way of controlling the game for me. There was not much that 
I did not like about this game, but it did annoy me that to start it you had to step into 
a glowing circle that would usually end up right behind you.
\section{Microsoft HoloLens}
\subsection{Holograms}
Microsoft's \textit{Holograms} program for the HoloLens was an excellent demo of what the 
device is capable of. The holograms tracked extremely well against the surrounding room,
and some actually seemed pretty realistic. Most of the holograms were not interactive,
which allowed this app to avoid a major problem that the HoloLens has: it is not very
responsive to input, and the input is uncomfortable. Over time, I got better at getting
the HoloLens to recognize my clicks, but holding your hand in the position needed to do so
gets tiring quickly.
\subsection{Microsoft Edge}
\section{Oculus Rift}
\subsection{Fruit Ninja}
\subsection{Ritchie's Plank Experience}
\end{document}
