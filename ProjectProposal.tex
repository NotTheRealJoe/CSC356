\documentclass[11pt]{article}
\usepackage[margin=0.79in]{geometry}
\usepackage{titlesec}
\titlespacing*{\section}{0pt}{0.5cm}{0pt}
\title{CSC356 Project Proposal\vspace{-2ex}}
\author{Joe Tortorello}
\date{\vspace{-2ex}23 March 2018\vspace{-2ex}}
%-----------------------------------------------------------------------------------------
\begin{document}
\maketitle
\section{Introduction}
The project I wish to propose is an augmented-reality mapping and navigation application
for the Microsoft HoloLens. The application would allow the user to specify a destination
in the physical world, and the HoloLens will display navigation instructions on the ground
in their respective physical locations. The project will take advantage of the HoloLens's
ability to differentiate between rooms, store room models persistently, and match stored
room models with the present room. The HoloLens's room modeling functionality will also be
used for the purpose of attaching user interface elements to the ground in front of the
user.
\section{Scale and Limitations}
Several issues I would anticipate this project to experience relate to the use of the
HoloLens outdoors. The size of room that the HoloLens can map is limited, and this limit
would certainly be exceeded in an outdoor environment. Additionally, the HoloLens does
not have a GPS sensor, so location data would have to be determined using a less accurate
location service that relies on distance of known Wi-Fi networks. Finally, the visual
performance of the HoloLens is poor in bright light, so on a sunny day the user interface
would not be very visible.

Due to these limitations, I am proposing a smaller scale project, where all navigation
will take place within the confines of Law Hall. The need for geolocation will be
eliminated by using room model matching to identify where the user is within the
building. This means that areas the user might enter would need to be pre-mapped by the
HoloLens, but this is doable on the scale of a single building. Instructions for moving
from one room to another could be stored within the HoloLens at this scale, but could also
be offloaded to an external server since the HoloLens will have a reasonably stable
network connection when operating only within the building.

\section{Future Applications}
There are several future applications that add relevance to this project. At the scale of
a single building, the application could be used within retail stores, to help customers
find what they are looking for within the store. A similar approach could be used in
buildings such as shopping malls, to help users find a particular retailer within the
mall, or to find other facilities such as restrooms. In facilities with a parking garage,
the location of the user's vehicle could potentially be stored to make it easier to find
later.

On a larger scale, an application like this one could, of course, be used for navigation
in vehicles. In vehicles, it might make more sense for the user interface to be
implemented by a windshield projection rather than a head-mounted display, but the concept
would remain the same. This technology could have initial applications for public safety,
allowing routes for public safety vehicles to be determined based upon traffic data by a
computer, and presented to the driver in a clear way, potentially improving the speed at
which these vehicles could get to a certain location. Of course, applications could also
extend to the consumer space, providing driving directions much like a smartphone
navigation application today, but presenting them with clear visual cues that do not
require the driver to look down or to the side to see. Finally, more advanced navigational
information, such as coordinate grids and other positioning data may be useful to ship or
aircraft operators. For aircraft specifically, the application could allow runways to be
seen during descent before the aircraft drops below clouds, or through fog of any density.

\end{document}